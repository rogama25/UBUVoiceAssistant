\apendice{Especificación de Requisitos}

\section{Introducción}
En este apéndice se explican los objetivos del proyecto y los requisitos que se han definido para llevarlo a cabo.

\section{Objetivos generales}
El principal objetivo del proyecto es mejorar la versión anterior del asistente del UBUVoiceAssistant. Para cumplirlo, se proponen algunas mejoras, como una mejora en el proceso de instalación y configuración inicial o en la interacción con el asistente. También se propone conseguir una mayor robustez en la aplicación, obtener más información de Moodle e incluso poder enviar datos a la plataforma.

\section{Catalogo de requisitos}
Los nuevos requisitos que se han añadido a la aplicación son los siguientes:
\subsection{Requisitos funcionales}
\begin{itemize}
    \item RF-1 Obtener datos de Moodle: La aplicación debe poder obtener distintos datos de Moodle.
    \begin{itemize}
        \item RF-1.1 Obtener los mensajes privados: La aplicación debe poder leer los últimos mensajes que ha recibido el usuario de la misma.
        \item RF-1.2 Buscar en los foros: La aplicación debe poder buscar en los foros los hilos relacionados con una frase que diga el usuario.
    \end{itemize}
    \item RF-2 Enviar datos a Moodle: La aplicación debe poder enviar algunos datos a nuestra plataforma de Moodle.
    \begin{itemize}
        \item RF-2.1 Enviar mensajes privados: La aplicación debe poder enviar mensajes privados a otros usuarios con los que se tiene una conversación abierta o que comparten un curso con el usuario de la aplicación.
    \end{itemize}
    \item RF-3 Mostrar un sistema de ayuda: La aplicación debe poder indicar al usuario cómo interactuar, sugiriéndole algunas frases que puede decir.
    \item RF-4 Mejorar la instalación: Se deben mejorar varios aspectos para hacer más cómodo el proceso de puesta en marcha.
    \begin{itemize}
        \item RF-4.1 Crear un instalador: Se debe crear un programa que prepare todas las dependencias necesarias y coloque todos los archivos en las ubicaciones necesarias para el correcto funcionamiento de la aplicación.
        \item RF-4.2 Crear un asistente de emparejamiento: La aplicación debe guiar al usuario por el proceso de emparejamiento del cliente con el servidor de Mycroft.
    \end{itemize}
\end{itemize}

\subsection{Requisitos no funcionales}
\begin{itemize}
    \item RNF-1 Soporte para Windows: La aplicación debe poder ser ejecutada en Windows 10.
    \item RNF-2 Naturalidad: La interacción de la aplicación con el usuario debe ser lo más natural posible.
    \item RNF-3 Robustez: Se debe mejorar la robustez de la aplicación, haciéndola más tolerante a fallos.
    \item RNF-4 Internacionalización: La aplicación tiene que permitir añadir idiomas nuevos fácilmente.
    \item RNF-5 Calidad de código: La calidad del código debe ser buena, y tener una documentación completa.
\end{itemize}

\section{Especificación de requisitos}