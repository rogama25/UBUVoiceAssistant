\capitulo{3}{Conceptos teóricos}

En este apartado se desarrollan algunos de los principales conceptos teóricos que son necesarios para comprender el proyecto, como las definiciones de los componentes usados, o su funcionamiento.

\section{Asistente de voz}
Los asistentes de voz \cite{virtualassistant} son programas que permiten a un usuario interactuar con una máquina, intentando que la comunicación entre ellos sea lo más natural posible, como si estuviésemos hablando con otra persona. Habitualmente para esto se usa directamente la voz, aunque también se suelen poder introducir órdenes escritas que realizan las mismas funciones. El asistente procesa las órdenes y responde de una forma similar.

\section{Wake word}
Para poder distinguir si una persona está hablando a un asistente de voz o lo está haciendo por cualquier otro motivo, se suelen usar una serie de palabras que el usuario debe pronunciar para que el asistente comience a escuchar y procese la orden que se le da. Estas palabras son conocidas habitualmente como \textit{Wake word} \cite{wakeword} Cada asistente usa una \textit{wake word} diferente. En Mycroft es configurable, y para este en particular, se suele usar \textit{``Hey Mycroft''}

\section{Skill}
Las skills \cite{skills} son programas o aplicaciones que están diseñadas para un asistente de voz. Normalmente cada una de las \textit{skills} se encarga de una función diferente, por ejemplo, una podría encargarse de mostrarte los próximos eventos del calendario mientras que otra se encarga de leerte las últimas noticias. Cada \textit{skill} está compuesta por varios elementos, como pueden ser las \textit{utterances}, los \textit{intents}, los \textit{dialogs}, los \textit{prompts} o el contexto, que se describen más adelante.

\subsection{Utterance}
Una \textit{utterance} \cite{mycroftglossary} es la frase que dice el usuario que sirve para activar una \textit{skill} concreta y que en algunos causará que el asistente realice una acción determinada. Un posible ejemplo de \textit{utterance} relacionada con UBUVirtual sería: ``Dime los próximos eventos de Sistemas Distribuidos''

\subsection{Intent}
Un \textit{intent} \cite{mycroftglossary} son unas palabras clave del \textit{utterance} que permiten al asistente determinar cuál es la acción que el usuario quiere realizar. Una \textit{skill} puede tener asociados varios \textit{intents}. Varios de ellos pueden lanzar una misma acción, consiguiendo una interacción más natural, ya que el usuario puede pedir lo mismo de diferentes formas, por ejemplo: ``Abrir el calendario'' o ``Consultar los eventos''.

\subsection{Dialog}
Este término es específico de \textit{Mycroft}, aunque muchos otros asistentes usan otras herramientas para desempeñar la misma función. Los \textit{dialogs} son las frases con las que responde el asistente a las peticiones que hace el usuario. También es posible que en vez de un \textit{dialog}, la respuesta sea completamente dinámica y no se usen. En el ejemplo anterior, el \textit{dialog} podría ser ``Los próximos eventos son: Entrega de la práctica 2 para el viernes 30 de Abril''.

\subsection{Contexto}
El contexto es una herramienta que se usa cada vez más dentro de los asistentes de voz ya que sirve para guardar parte de la información que se ha intercambiado en las anteriores preguntas y hacer que en las próximas interacciones los resultados ofrecidos por el programa estén relacionados. Por ejemplo, podríamos decir primero ``Dime los foros de Sistemas Distribuidos'' y si luego decimos ``Crear un hilo en los foros de esa asignatura'', el programa asociará ``esa asignatura'' con ``Sistemas Distribuidos''.

\subsection{Prompt}
Los \textit{prompts} son preguntas que el asistente hace al usuario si el asistente necesita más información para completar la orden del usuario. Por ejemplo, si queremos crear un hilo en los foros de una asignatura, el asistente podría repetir el texto que ha entendido, y finalmente, usando un \textit{prompt}, preguntar al usuario si el texto que ha entendido es correcto o no.

\section{Fuzzy matching}
Debido a que los sistemas de reconocimiento de voz a día de hoy distan bastante de ser perfectos, en ocasiones no detectan correctamente las palabras que el usuario dijo. Esto suele ocurrir de manera más habitual en el caso de palabras que no están recogidas en un diccionario o son usadas de manera frecuente, como por ejemplo, los nombres propios. Si se tiene una lista de opciones entre las que el usuario tiene que elegir una, se puede usar \textit{fuzzy matching} \cite{fuzzymatching} para estimar cuál es la palabra más parecida de entre las opciones a lo que el usuario quiso decir. En nuestra aplicación, se pueden obtener los nombres de los contactos del usuario, estimar el parecido de la frase identificada por el reconocimiento de voz con cada uno de ellos y elegir el más similar.

\section{Speech-to-text}
El \textit{Speech-to-text}, también conocido como STT o reconocimiento de habla \cite{stt} es una disciplina que desarrolla una serie de tecnologías para interpretar y traducir el lenguaje hablado en texto. Habitualmente este tipo de tecnologías se conocen como ``Reconocimiento de voz'', pero este término no es correcto, ya que se usa para diferenciar la voz de una persona de la de otra.

El reconocimiento de voz puede ser o bien ajustado para una sola persona, o bien independiente del hablante. En el primero de los casos hay que realizar un entrenamiento previo, en el que la persona tiene que decir unas palabras concretas para que el modelo se ajuste a su voz concreta.

Esta tecnología se está usando en la actualidad en numerosos interfaces de voz diferentes, como en sistemas de marcación telefónica, en domótica, en sistemas preparados para personas discapacitadas o bien en dictado de textos.

\section{Text-to-speech}
El \textit{Text-to-speech}, también conocido como TTS o síntesis de voz \cite{tts} es una tecnología que permite generar habla a partir de textos. En un primer momento esto se realizaba en equipos específicamente diseñados para ello, aunque en la actualidad se hace por software en multitud de dispositivos de ámbito general.

Esto se puede conseguir bien a través de sonidos pregrabados que se unen en el momento de generar las frases, o bien se crea un modelo de las características del habla humana, permitiendo generar una voz completamente sintética.

También es importante destacar que suele estar formado por dos partes. La primera de ellas convierte abreviaciones o números que pueda haber en el texto a palabras completas, tal y como las diríamos, y les asigna los fonemas que le corresponden a cada una de las letras. La segunda parte se encarga de la generación de los sonidos a partir de los fonemas anteriormente generados.