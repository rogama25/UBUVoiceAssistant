\capitulo{7}{Conclusiones y Líneas de trabajo futuras}

\section{Conclusiones}


\section{Líneas de trabajo futuras}
En este apartado se incluyen algunas de las ideas que surgieron en la anterior versión del proyecto que no se han podido cumplir, junto a otras  nuevas que han ido surgiendo a lo largo del desarrollo.

Por el momento sólo es posible usar la aplicación en Linux o en Windows 10 mediante el WSL, pero no se puede usar en MacOS ni en móviles, donde quizá es más útil. Por el momento es necesario esperar a que avance el desarrollo de Mycroft para que sea viable usarlo en esas plataformas. También se podría optar por intentar conseguir una funcionalidad similar mediante otras tecnologías que sí sean multiplataforma, o crear una aplicación cliente ligero que se conecte a un servidor de la universidad donde se procesen todas las peticiones de los diferentes usuarios.

También podría ser interesante la posibilidad de trabajar sin conexión. Sería necesario descargar los datos obtenidos de Moodle a una serie de archivos o una base de datos local que se podrían consultar más adelante cuando no haya una conexión disponible.