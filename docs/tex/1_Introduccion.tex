\capitulo{1}{Introducción}

Los asistentes de voz son una tecnología nueva que no se ha usado apenas hasta el momento para la educación. Sin embargo, se ha visto que puede ser muy interesante usarlas, debido a que es mucho más rápido y cómodo buscar determinadas informaciones con un comando de voz en vez de navegar por diferentes menús que pueden ser bastante complejos.

Partiendo de una versión inicial del proyecto, se busca simplificar la instalación del proyecto, añadir funcionalidades nuevas, como la posibilidad de consultar los mensajes privados de Moodle o la posibilidad de usarlo desde un móvil, y simplificar la interacción con el usuario.

En esta memoria se irán explicando los conceptos necesarios para entenderlo, así como las tecnologías usadas para realizar el proyecto.
