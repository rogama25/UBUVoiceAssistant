\capitulo{1}{Introducción}

Los asistentes de voz son una tecnología nueva que no se ha usado apenas hasta el momento para la educación. Sin embargo, se ha visto que puede ser muy interesante usarlas, debido a que es mucho más rápido y cómodo buscar determinadas informaciones con un comando de voz en vez de navegar por diferentes menús que pueden ser bastante complejos.

Este proyecto se basa en mejorar la primera versión de un asistente de voz que permitía conectarte a una plataforma \textit{Moodle} (como UBUVirtual). Con respecto a la anterior versión, se busca simplificar el proceso de instalación y configuración inicial, conseguir una apariencia más moderna, incluir una funcionalidad que te permita consultar y enviar los mensajes privados, ampliar el sistema de ayuda y hacer que la conversación sea más natural, haciendo que el programa vaya almacenando el contexto.

En esta memoria se irán explicando los conceptos necesarios para su comprensión, así como las tecnologías usadas para realizar el proyecto, los problemas que se han encontrado a lo largo del desarrollo, las decisiones tomadas y las soluciones para intentar solventarlos.
