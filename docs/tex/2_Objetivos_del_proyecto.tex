\capitulo{2}{Objetivos del proyecto}

Este apartado explica de forma precisa y concisa cuales son los objetivos que se persiguen con la realización del proyecto. Se puede distinguir entre los objetivos marcados por los requisitos del software a construir y los objetivos de carácter técnico que plantea a la hora de llevar a la práctica el proyecto.

\section*{Objetivos generales}
\begin{itemize}
    \item Mejorar la instalación del proyecto.
    \item Facilitar la internacionalización.
    \item Añadir sistemas de log.
    \item Mejorar el estilo visual.
    \item Conseguir mayor robustez y estabilidad en la aplicación.
    \item Mejorar la experiencia de usuario con una interacción más natural.
    \item Ampliar funcionalidad de interacción con Moodle
\end{itemize}

\section*{Objetivos técnicos}
\begin{itemize}
    \item Crear de un script de instalación, que sea capaz de descargar e instalar todos los componentes necesarios, y de poder desinstalar el programa, o poder actualizar a una nueva versión.
    \item Construir nuevas interfaces gráficas
    \item Usar HTML embebido en la aplicación para hacer que sea más visual que con los widgets del sistema operativo.
    \item Incorporar una biblioteca de internacionalización para incluir nuevos idiomas de manera más sencilla, y sin tener que modificar el código.
    \item Crear una skill de Mycroft que aproveche el contexto para recordar las anteriores interacciones.
    \item Aprovechar el fuzzy matching para poder entender lo que el usuario quiere decir aunque el reconocimiento de voz no entienda bien las palabras.
\end{itemize}
