\apendice{Documentación de usuario}

\section{Introducción}
En este apartado se explica cómo se puede instalar y usar el proyecto.

\section{Requisitos de usuarios}
\begin{itemize}
    \item Sistema operativo Windows 10 u Ubuntu 20.04 o superior.
    \item Conexión a internet
    \item 3GB de memoria RAM
    \item 5GB de espacio en disco libres
\end{itemize}

\section{Instalación}

\subsection{Windows Susbsystem for Linux}
En el caso de que hayamos optado por usar Windows 10 como sistema operativo, es necesario que instalemos este sistema para poder ejecutar la aplicación. Si se va a usar Ubuntu, saltar a la siguiente sección.

\subsubsection{Instalación de WSL}
Para comenzar, deberemos conocer la versión de Windows 10 que estamos ejecutando en nuestro equipo. Para ello, abrimos el menú de configuración de Windows, hacemos click en Sistema'' y luego en Acerca de''. En esa pantalla deberemos fijarnos que tengamos un sistema operativo de 64 bits y tengamos la versión 1607 o superior.

Ahora, tenemos que buscar Powershell'' en el menú de inicio y la ejecutaremos como administrador. En esa ventana, escribimos el siguiente comando:

\texttt{dism.exe /online /enable-feature \`{}\newline
/featurename:Microsoft-Windows-Subsystem-Linux /all /norestart}

En el caso de que tengamos la versión 1903 o superior, con la compilación 18362 o superior, es recomendable instalar WSL2 para obtener un mayor rendimiento. En el caso de no tenerlo, saltamos al apartado de la instalación de Ubuntu para WSL.

Lo siguiente es activar la virtualización, usando este comando en la Powershell que habíamos abierto:

\texttt{dism.exe /online /enable-feature \`{}\newline
/featurename:VirtualMachinePlatform /all /norestart}

En este punto, reiniciamos el ordenador. Después del reinicio, instalamos el paquete de actualización del kernel de Linux, desde \href{https://wslstorestorage.blob.core.windows.netwslblob/wsl_update_x64.msi}{este enlace}.

Abrimos otra Powershell como administrador y establecemos el uso de WSL2 por defecto:

\texttt{wsl --set-default-version 2}

\subsubsection{Instalación de Ubuntu para WSL}
Para instalar Ubuntu en el Windows Subsystem for Linux, tenemos que descargarlo desde la Microsoft Store, accesible en \href{https://www.microsoft.com/es-es/p/ubuntu/9nblggh4msv6}{este enlace}. Una vez instalado, es probable que se abra de manera automática una terminal en la que se nos pedirá elegir un nombre de usuario y una contraseña. En el caso de que no se abra automáticamente, podremos abrirla desde el menú de inicio, en el acceso directo a Ubuntu.

\subsubsection{Instalación de Pulseaudio y servidor X11}
Para que el susbsistema pueda recibir sonido del micrófono y mostrar interfaces gráficas debemos instalar lo siguiente:

VcXserv, descargable desde \href{https://sourceforge.net/projects/vcxsrv/}{este enlace}. Deberemos añadir al .bashrc del WSL las siguientes líneas:

\texttt{export DISPLAY=:0} (únicamente si no usamos WSL2)

\texttt{export DISPLAY=\$(awk '/nameserver / \{print \$2; exit\}' \textbackslash\newline/etc/resolv.conf 2>/dev/null):0} (si usamos WSL2)

\texttt{export LIBGL\_ALWAYS\_INDIRECT=1}

Pulseaudio, descargable desde \href{https://www.freedesktop.org/wiki/Software/PulseAudio/Ports/Windows/Support/}{este enlace}. Tendremos que editar los siguientes archivos:

\texttt{etc/pulse/default.pa}, donde tendremos que añadir:

\texttt{load-module module-native-protocol-tcp auth-ip-acl=127.0.0.1}

\texttt{etc/pulse/daemon.conf}, donde cambiaremos el \texttt{exit-idle-time} a -1.

Ejecutamos el programa bin/pulseaudio.exe y añadimos la siguiente línea al .bashrc:
\texttt{export PULSE\_SERVER=tcp:127.0.0.1}

\subsection{Descarga del repositorio}
Para descargar el código de la aplicación se puede hacer usando un navegador o usando la terminal. En el caso de que estemos usando Windows Subsystem for Linux, es recomendable usar la terminal.

\subsubsection{Usando un navegador web}
Desde cualquier navegador de internet, hay que ir a la \href{https://github.com/rogama25/UBUVoiceAssistant}{página web del repositorio}.

Allí, en la parte derecha, haz click en la sección ``Releases'' y, en el desplegable de ``Assets'', descarga el ``Source code (.zip)'' que aparezca en la versión más reciente.

Una vez se complete la descarga, primero hay que descomprimirlo y luego abrir una terminal donde nos moveremos hasta la carpeta que se ha creado con el nombre de la versión usando el comando \texttt{cd}.

Allí hay que ejecutar el comando ``sudo ./install.sh install'' y esperar a que termine. Puede que tarde varios minutos, dependiendo de la velocidad de internet y del equipo, ya que descarga y configura todos los componentes necesarios para ejecutar el programa.

\subsubsection{Usando la terminal}
Desde una terminal, ejecutamos los siguientes comandos:
\begin{itemize}
    \item \texttt{sudo apt install git}
    \item \texttt{git clone https://github.com/rogama25/UBUVoiceAssistant.git}
    \item \texttt{cd UBUVoiceAssistant}
    \item \texttt{sudo ./install.sh install}
\end{itemize}

\section{Manual del usuario}
Después de haber completado la instalación, en el caso de que estemos usando Ubuntu como sistema operativo, tendremos un icono en su lanzador con el que podremos abrir nuestro programa. Si estamos usando el Windows Susbsystem for Linux o si no aparece el icono en el lanzador de aplicaciones, deberemos escribir \texttt{UBUVoiceAssistant} para ejecutar el programa.

La primera ventana que veremos es la de inicio de sesión. En la esquina superior derecha podemos elegir el idioma de la aplicación y en la parte inferior tendremos los campos donde debemos introducir los credenciales de Moodle. Deberemos especificar también la dirección web del servidor, siendo en nuestro caso \texttt{https://ubuvirtual.ubu.es}. Las direcciones deben incluir el protocolo http o https.

Una vez hayamos iniciado sesión por primera vez (puede tardar varios minutos mientras Mycroft instala algunos de sus componentes), se nos mostrará la ventana de emparejamiento. En ella aparecen los pasos que debemos realizar para vincular el cliente de Mycroft que se ha instalado en el equipo con sus servicios web. El procedimiento es el siguiente:
\begin{itemize}
    \item Abrir en un navegador de Internet la \href{https://mycroft.ai}{página web de Mycroft}.
    \item Registrarnos o iniciar sesión en una cuenta que ya tengamos.
    \item En la parte superior derecha, donde aparece el icono del perfil, hacer click en ``Add Device''.
    \item Escribir el código dicho por Mycroft en el campo que pone ``Pairing Code''.
    \item En la parte inferior, seleccionar ``Google Voice'' y guardar los cambios.
\end{itemize}

Pasados unos segundos, se debería abrir la ventana con la interfaz de chat. En ella se muestra la conversación que tengamos con el asistente de voz. A partir de este punto podremos decir nuestras órdenes a través del micrófono. Al decir ``Hey Mycroft'', deberíamos oír un sonido que significa que ha detectado la wake word y está grabando la siguiente frase que digamos. También podemos silenciar el micrófono mediante el botón que hay en la interfaz, o escribir nuestras órdenes por teclado en el campo inferior. Haciendo click en el botón de la parte superior derecha de la ventana, podemos activar y desactivar las skills que queramos.

-- Faltan imágenes --