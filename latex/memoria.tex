\documentclass[a4paper,12pt,twoside]{memoir}

% Castellano
\usepackage[spanish,es-tabla]{babel}
\selectlanguage{spanish}
\usepackage[utf8]{inputenc}
\usepackage[T1]{fontenc}
\usepackage{lmodern} % Scalable font
\usepackage{microtype}
\usepackage{placeins}

\usepackage[backend=bibtex,dateabbrev=false]{biblatex}
\addbibresource{bibliografia.bib}

\RequirePackage{booktabs}
\RequirePackage[table]{xcolor}
\RequirePackage{xtab}
\RequirePackage{multirow}

% Links
\PassOptionsToPackage{hyphens}{url}\usepackage[colorlinks]{hyperref}
\hypersetup{
	allcolors = {red}
}

% Ecuaciones
\usepackage{amsmath}

% Rutas de fichero / paquete
\newcommand{\ruta}[1]{{\sffamily #1}}

% Párrafos
\nonzeroparskip

% Huérfanas y viudas
\widowpenalty100000
\clubpenalty100000

% Imagenes
\usepackage{graphicx}
\newcommand{\imagen}[2]{
	\begin{figure}[!h]
		\centering
		\includegraphics[width=0.9\textwidth]{#1}
		\caption{#2}\label{fig:#1}
	\end{figure}
	\FloatBarrier
}

\newcommand{\imagenflotante}[2]{
	\begin{figure}%[!h]
		\centering
		\includegraphics[width=0.9\textwidth]{#1}
		\caption{#2}\label{fig:#1}
	\end{figure}
}



% El comando \figura nos permite insertar figuras comodamente, y utilizando
% siempre el mismo formato. Los parametros son:
% 1 -> Porcentaje del ancho de página que ocupará la figura (de 0 a 1)
% 2 --> Fichero de la imagen
% 3 --> Texto a pie de imagen
% 4 --> Etiqueta (label) para referencias
% 5 --> Opciones que queramos pasarle al \includegraphics
% 6 --> Opciones de posicionamiento a pasarle a \begin{figure}
\newcommand{\figuraConPosicion}[6]{%
  \setlength{\anchoFloat}{#1\textwidth}%
  \addtolength{\anchoFloat}{-4\fboxsep}%
  \setlength{\anchoFigura}{\anchoFloat}%
  \begin{figure}[#6]
    \begin{center}%
      \Ovalbox{%
        \begin{minipage}{\anchoFloat}%
          \begin{center}%
            \includegraphics[width=\anchoFigura,#5]{#2}%
            \caption{#3}%
            \label{#4}%
          \end{center}%
        \end{minipage}
      }%
    \end{center}%
  \end{figure}%
}

%
% Comando para incluir imágenes en formato apaisado (sin marco).
\newcommand{\figuraApaisadaSinMarco}[5]{%
  \begin{figure}%
    \begin{center}%
    \includegraphics[angle=90,height=#1\textheight,#5]{#2}%
    \caption{#3}%
    \label{#4}%
    \end{center}%
  \end{figure}%
}
% Para las tablas
\newcommand{\otoprule}{\midrule [\heavyrulewidth]}
%
% Nuevo comando para tablas pequeñas (menos de una página).
\newcommand{\tablaSmall}[5]{%
 \begin{table}
  \begin{center}
   \rowcolors {2}{gray!35}{}
   \begin{tabular}{#2}
    \toprule
    #4
    \otoprule
    #5
    \bottomrule
   \end{tabular}
   \caption{#1}
   \label{tabla:#3}
  \end{center}
 \end{table}
}

%
% Nuevo comando para tablas pequeñas (menos de una página).
\newcommand{\tablaSmallSinColores}[5]{%
 \begin{table}[H]
  \begin{center}
   \begin{tabular}{#2}
    \toprule
    #4
    \otoprule
    #5
    \bottomrule
   \end{tabular}
   \caption{#1}
   \label{tabla:#3}
  \end{center}
 \end{table}
}

\newcommand{\tablaApaisadaSmall}[5]{%
\begin{landscape}
  \begin{table}
   \begin{center}
    \rowcolors {2}{gray!35}{}
    \begin{tabular}{#2}
     \toprule
     #4
     \otoprule
     #5
     \bottomrule
    \end{tabular}
    \caption{#1}
    \label{tabla:#3}
   \end{center}
  \end{table}
\end{landscape}
}

%
% Nuevo comando para tablas grandes con cabecera y filas alternas coloreadas en gris.
\newcommand{\tabla}[6]{%
  \begin{center}
    \tablefirsthead{
      \toprule
      #5
      \otoprule
    }
    \tablehead{
      \multicolumn{#3}{l}{\small\sl continúa desde la página anterior}\\
      \toprule
      #5
      \otoprule
    }
    \tabletail{
      \hline
      \multicolumn{#3}{r}{\small\sl continúa en la página siguiente}\\
    }
    \tablelasttail{
      \hline
    }
    \bottomcaption{#1}
    \rowcolors {2}{gray!35}{}
    \begin{xtabular}{#2}
      #6
      \bottomrule
    \end{xtabular}
    \label{tabla:#4}
  \end{center}
}

%
% Nuevo comando para tablas grandes con cabecera.
\newcommand{\tablaSinColores}[6]{%
  \begin{center}
    \tablefirsthead{
      \toprule
      #5
      \otoprule
    }
    \tablehead{
      \multicolumn{#3}{l}{\small\sl continúa desde la página anterior}\\
      \toprule
      #5
      \otoprule
    }
    \tabletail{
      \hline
      \multicolumn{#3}{r}{\small\sl continúa en la página siguiente}\\
    }
    \tablelasttail{
      \hline
    }
    \bottomcaption{#1}
    \begin{xtabular}{#2}
      #6
      \bottomrule
    \end{xtabular}
    \label{tabla:#4}
  \end{center}
}

%
% Nuevo comando para tablas grandes sin cabecera.
\newcommand{\tablaSinCabecera}[5]{%
  \begin{center}
    \tablefirsthead{
      \toprule
    }
    \tablehead{
      \multicolumn{#3}{l}{\small\sl continúa desde la página anterior}\\
      \hline
    }
    \tabletail{
      \hline
      \multicolumn{#3}{r}{\small\sl continúa en la página siguiente}\\
    }
    \tablelasttail{
      \hline
    }
    \bottomcaption{#1}
  \begin{xtabular}{#2}
    #5
   \bottomrule
  \end{xtabular}
  \label{tabla:#4}
  \end{center}
}



\definecolor{cgoLight}{HTML}{EEEEEE}
\definecolor{cgoExtralight}{HTML}{FFFFFF}

%
% Nuevo comando para tablas grandes sin cabecera.
\newcommand{\tablaSinCabeceraConBandas}[5]{%
  \begin{center}
    \tablefirsthead{
      \toprule
    }
    \tablehead{
      \multicolumn{#3}{l}{\small\sl continúa desde la página anterior}\\
      \hline
    }
    \tabletail{
      \hline
      \multicolumn{#3}{r}{\small\sl continúa en la página siguiente}\\
    }
    \tablelasttail{
      \hline
    }
    \bottomcaption{#1}
    \rowcolors[]{1}{cgoExtralight}{cgoLight}

  \begin{xtabular}{#2}
    #5
   \bottomrule
  \end{xtabular}
  \label{tabla:#4}
  \end{center}
}


















\graphicspath{ {./img/} }

% Capítulos
\chapterstyle{bianchi}
\newcommand{\capitulo}[2]{
	\setcounter{chapter}{#1}
	\setcounter{section}{0}
	\chapter*{#2}
	\addcontentsline{toc}{chapter}{#2}
	\markboth{#2}{#2}
}

% Apéndices
\renewcommand{\appendixname}{Apéndice}
\renewcommand*\cftappendixname{\appendixname}

\newcommand{\apendice}[1]{
	%\renewcommand{\thechapter}{A}
	\chapter{#1}
}

\renewcommand*\cftappendixname{\appendixname\ }

% Formato de portada
\makeatletter
\usepackage{xcolor}
\newcommand{\tutor}[1]{\def\@tutor{#1}}
\newcommand{\course}[1]{\def\@course{#1}}
\definecolor{cpardoBox}{HTML}{E6E6FF}
\def\maketitle{
  \null
  \thispagestyle{empty}
  % Cabecera ----------------
\noindent\includegraphics[width=\textwidth]{cabecera}\vspace{1cm}%
  \vfill
  % Título proyecto y escudo informática ----------------
  \colorbox{cpardoBox}{%
    \begin{minipage}{.8\textwidth}
      \vspace{.5cm}\Large
      \begin{center}
      \textbf{TFG del Grado en Ingeniería Informática}\vspace{.6cm}\\
      \textbf{\LARGE\@title{}}
      \end{center}
      \vspace{.2cm}
    \end{minipage}

  }%
  \hfill\begin{minipage}{.20\textwidth}
    \includegraphics[width=\textwidth]{escudoInfor}
  \end{minipage}
  \vfill
  % Datos de alumno, curso y tutores ------------------
  \begin{center}%
  {%
    \noindent\LARGE
    Presentado por \@author{}\\ 
    en Universidad de Burgos --- \@date{}\\
    Tutor: \@tutor{}\\
  }%
  \end{center}%
  \null
  \cleardoublepage
  }
\makeatother

\newcommand{\nombre}{Álvaro Delgado Pascual} %%% cambio de comando

% Datos de portada
\title{UBUAssistant\\Interacción por voz con la plataforma Moodle}
\author{\nombre}
\tutor{D. Raúl Marticorena Sanchez}
\date{\today}

\begin{document}

\maketitle


\newpage\null\thispagestyle{empty}\newpage


%%%%%%%%%%%%%%%%%%%%%%%%%%%%%%%%%%%%%%%%%%%%%%%%%%%%%%%%%%%%%%%%%%%%%%%%%%%%%%%%%%%%%%%%
\thispagestyle{empty}


\noindent\includegraphics[width=\textwidth]{cabecera}\vspace{1cm}

\noindent D. Raúl Marticorena Sánchez, profesor del departamento de Ingeniería Informática, área de Lenguajes y Sistemas Informáticos.

\noindent Expone:

\noindent Que el alumno D. \nombre, con DNI 71363793Z, ha realizado el Trabajo final de Grado en Ingeniería Informática titulado UBUAssistant. 

\noindent Y que dicho trabajo ha sido realizado por el alumno bajo la dirección del que suscribe, en virtud de lo cual se autoriza su presentación y defensa.

\begin{center} %\large
En Burgos, {\large \today}
\end{center}

\vfill\vfill\vfill

% para casos con solo un tutor comentar lo anterior
% y descomentar lo siguiente
Vº. Bº. del Tutor:\\[2cm]
D. Raúl Marticorena Sánchez


\newpage\null\thispagestyle{empty}\newpage




\frontmatter

% Abstract en castellano
\renewcommand*\abstractname{Resumen}
\begin{abstract}
Los asistentes de voz son una nueva tecnología cuyo uso se ha disparado durante los útlimos años. Estos asistentes nos permiten realizar distintas tareas de una forma muy cómoda, interactiva y empleando poco tiempo. Tareas como consultar el tiempo, una búsqueda en Internet, la compra... en definitiva una gran variedad de tareas, en muchos campos distintos.

En el ámbito de la educación, la enseñanza \textit{online} y las aulas virtuales también han tenido un importante desarrollo tecnológico, siendo Moodle un referente en este campo. Sin embargo, los asistentes de voz no se han utilizado en este área, a pesar de sus características y posibilidades que ofrecen para los estudiantes. Es por ello que ha surgido la idea de este proyecto, para dar a los usuarios de estas plataformas de educación una nueva forma de interactuar con ellas, de una forma más rápida y cómoda.

UBUAssistant es una aplicación cliente desarrollada en Python utilizando el asistente de voz Mycroft. Esta aplicación permite que el usuario, mediante comandos de voz o a través de texto, obtenga información sobre el curso académico sin necesidad de conocer como funciona Moodle ni saber cómo tiene que navegar por la plataforma para obtener esta información.
\end{abstract}

\renewcommand*\abstractname{Descriptores}
\begin{abstract}
Asistente de voz, Moodle, REST, intent, utterance, TTS, STT, skill
\end{abstract}

\clearpage

% Abstract en inglés
\renewcommand*\abstractname{Abstract}
\begin{abstract}
Voice assistants are a new technology whose use has been shot over during the last few years. These assistants allow us to perform varying tasks in an interactive, very comfortable way. Tasks such as check the weather, an Internet search, go shopping... in short a wide variety of tasks in many of fields.

In the educational enviroment, online teaching and virtual classrooms have had an important technological development too, Moodle being a referrer in this field. However, voice assistants have not been used in this area, despite of its features and the posibilities they bring to students. This is why that the idea of the project was conceived, to bring the users of these educational platforms a new way to interact with them, in a quicker and simpler way.

UBUAssistant is a client application developed in Python using Mycroft as the voice assistant. This application allows the user to obtain information of the academic course through voice commands or text input, without the need of knowing what is Moodle or what steps he needs to do in order to retrieve this information.
\end{abstract}

\renewcommand*\abstractname{Keywords}
\begin{abstract}
Voice assistant, Moodle, REST, intent, utterance, TTS, STT, skill
\end{abstract}

\clearpage

% Indices
\tableofcontents

\clearpage

\listoffigures

\clearpage

\mainmatter
\capitulo{1}{Introducción}

Los asistentes de voz son una tecnología nueva que no se ha usado apenas hasta el momento para la educación. Sin embargo, se ha visto que puede ser muy interesante usarlas, debido a que es mucho más rápido y cómodo buscar determinadas informaciones con un comando de voz en vez de navegar por diferentes menús que pueden ser bastante complejos.

Partiendo de una versión inicial del proyecto, se busca simplificar la instalación del proyecto, añadir funcionalidades nuevas, como la posibilidad de consultar los mensajes privados de Moodle o la posibilidad de usarlo desde un móvil, y simplificar la interacción con el usuario.

En esta memoria se irán explicando los conceptos necesarios para entenderlo, así como las tecnologías usadas para realizar el proyecto.

\capitulo{2}{Objetivos del proyecto}

Este apartado explica de forma precisa y concisa cuales son los objetivos que se persiguen con la realización del proyecto. Se puede distinguir entre los objetivos marcados por los requisitos del software a construir y los objetivos de carácter técnico que plantea a la hora de llevar a la práctica el proyecto.

\section*{Objetivos generales}
\begin{itemize}
    \item Mejorar la instalación del proyecto.
    \item Facilitar la internacionalización.
    \item Añadir sistemas de log.
    \item Mejorar el estilo visual.
    \item Añadir otros tipos de interacción.
\end{itemize}

\section*{Objetivos técnicos}
\begin{itemize}
    \item Aprendizaje y uso de contenedores Docker.
    \item Uso de APIs rest.
    \item Uso de skills con contexto.
    
\end{itemize}

\capitulo{3}{Conceptos teóricos}

En este apartado se desarrollan algunos de los principales conceptos teóricos que son necesarios para comprender el proyecto, como las definiciones de los componentes usados, o su funcionamiento.

\section{Asistente de voz}
Los asistentes de voz son programas que permiten a un usuario interactuar con una máquina, intentando que la comunicación entre ellos sea lo más natural posible, como si estuviésemos hablando con otra persona. Habitualmente para esto se usa directamente la voz, aunque también se suelen poder introducir órdenes escritas que realizan las mismas funciones. El asistente procesa las órdenes y responde de una forma similar.

\section{Wake word}
Para poder distinguir si una persona está hablando a un asistente de voz o lo está haciendo por cualquier otro motivo, se suelen usar una serie de palabras que el usuario debe pronunciar para que el asistente comience a escuchar y procese la orden que se le da. En este asistente es configurable, pero se suele usar \textit{``Hey Mycroft''}

\section{Skill}
Las skills son programas o aplicaciones que están diseñadas para un asistente de voz. Normalmente cada una de las skills se encarga de una función diferente, por ejemplo, una podría encargarse de mostrarte los próximos eventos del calendario mientras que otra se encarga de leerte las últimas noticias. Cada skill está compuesta por varios elementos, como pueden ser las utterance, los intents, los dialogs, los prompts o el contexto, que se describen más adelante.

\subsection{Utterance}
Una utterance es la frase que dice el usuario que sirve para activar una skill concreta y que en algunos causará que el asistente realice una acción determinada. Un posible ejemplo de utterance sería: ``Dime los próximos eventos de Sistemas Distribuidos''

\subsection{Intent}
Un intent son unas palabras clave del utterance que permiten al asistente determinar cuál es la acción que el usuario quiere realizar. Una skill puede tener asociados varios intents. Varios de ellos pueden lanzar una misma acción, consiguiendo una interacción más natural, ya que el usuario puede pedir lo mismo de diferentes formas, por ejemplo: ``Abrir el calendario'' o ``Consultar los eventos''.

\subsection{Dialog}
Este término es específico de Mycroft, aunque muchos otros asistentes usan otras herramientas para desempeñar la misma función. Los dialogs son las frases con las que responde el asistente a las peticiones que hace el usuario. También es posible que en vez de un dialog, la respuesta sea completamente dinámica y no se usen. En el ejemplo anterior, el dialog podría ser ``Los próximos eventos son: Entrega de la práctica 2 para el viernes 30 de Abril''.

\subsection{Contexto}
El contexto es una herramienta que se usa cada vez más dentro de los asistentes de voz ya que sirve para guardar parte de la información que se ha intercambiado en las anteriores preguntas y hacer que en las próximas interacciones los resultados ofrecidos por el programa estén relacionados. Por ejemplo, podríamos decir primero ``Dime los foros de Sistemas Distribuidos'' y si luego decimos ``Crear un hilo en los foros de esa asignatura'', el programa asociará ``esa asignatura'' con ``Sistemas Distribuidos''.

\subsection{Prompt}
Los prompts son preguntas que el asistente hace al usuario si el asistente necesita más información para completar la orden del usuario. Por ejemplo, si queremos crear un hilo en los foros de una asignatura, el asistente podría repetir el texto que ha entendido, y finalmente, usando un prompt, preguntar al usuario si el texto que ha entendido es correcto o no.

\section{Fuzzy matching}
Debido a que los sistemas de reconocimiento de voz a día de hoy distan bastante de ser perfectos, en ocasiones no detectan correctamente las palabras que el usuario dijo. Esto suele ocurrir de manera más habitual en el caso de palabras que no están recogidas en un diccionario o son usadas de manera frecuente, como por ejemplo, los nombres propios. Si se tiene una lista de opciones entre las que el usuario tiene que elegir una, se puede usar fuzzy matching para estimar cuál es la palabra más parecida de entre las opciones a lo que el usuario quiso decir. En nuestra aplicación, se pueden obtener los nombres de los contactos del usuario, estimar el parecido de la frase identificada por el reconocimiento de voz con cada uno de ellos y elegir el más similar.
\capitulo{4}{Técnicas y herramientas}

\section{Metodología de desarrollo}
Para el desarrollo del proyecto se ha utilizado metodología ágil, en concreto SCRUM. Se ha adoptado una estrategia de desarrollo iterativa, con sprints bimensuales en los que se han realizado las entregas parciales. Las reuniones de estado del proyecto se han realizado semanalmente, a diferencia de lo normal en SCRUM que se realizan diariamente.
\section{Moodle}
\href{https://moodle.org/}{Moodle} es una plataforma de aprendizaje que permite a docentes y alumnos impartir y recibir clases a distancia, así como facilitar la gestión de los cursos. Es una herramienta de código abierto y por lo tanto cualquiera puede utilizarlo. En su página se pueden encontrar versiones de prueba como \href{https://school.moodledemo.net/}{Mount Orange School}
\section{PyQt5}
\href{https://doc.qt.io/qtforpython/}{PyQt5} es un binding (una adaptación de una biblioteca para que sea usada en otro lenguaje). Qt es un framework multiplataforma orientado a objetos para desarrollar programas que utilizan interfaces gráficas de usuario. Es software libre y de código abierto.
\section{requests}
\href{https://requests.readthedocs.io/es/latest/}{Requests} es una biblioteca para Python que permite enviar peticiones HTTP de forma muy sencilla.
\section{MiKTeX}
\section{TeXstudio}
\capitulo{5}{Aspectos relevantes del desarrollo del proyecto}

Este apartado pretende comentar los aspectos importantes y problemas que han surgido con la realización del proyecto, así como las decisiones de añadir o no más funcionalidad a la aplciación.
\section{Utilización de AmazonWebServices y Alexa para la realización del proyecto}
La idea inicial era utilizar los servicios web de Amazon (Amazon Lambda) y la consola de desarrollador (Amazon Developer) para crear la Skill de Alexa y hostearla en sus servidores.

Ya que nunca había usado nunca estos servicios ni había desarrollado ninguna Skill o aplicación asistente de voz similar, leyendo la documentación de Alexa encontré una guía de cómo crear tu primera skill.
\subsection{Amazon Web Services}
Los Amazon Web Services, AWS a partir de ahora, es un conjunto de servicios de computación en la nube. En concreto, de todos estos servicios se utilizó Amazon Lambda, que es una plataforma sin servidor basada en eventos, que es la base de los AWS. Lambda ejecuta código como respuesta a eventos y gestiona los recursos necesitados por el código ejecutado.
\subsection{Amazon Alexa}
Es un servicio de voz en la nube para dispositivos de Amazon y dispositivos de terceros que usan Alexa.
\section{Problemas con Alexa}
\section{Mycroft como alternativa}
\capitulo{6}{Trabajos relacionados}

Si bien es cierto que este campo estaba prácticamente inexplorado hace unos pocos años, cada poco tiempo aparece algún proyecto que pone a prueba las capacidades de los asistentes en el campo de la educación. Algunos de los más interesantes son los siguientes:

En primer lugar tenemos los predecesores del proyecto actual, realizados por la Universidad de Burgos. Podemos encontrar tanto \href{https://github.com/adp1002/UBUVoiceAssistant}{la versión anterior de este mismo proyecto} como \href{https://github.com/cgc0045/TFG-UBUassistant}{una aplicación para Android} que permite hacer preguntas sobre la información de la página web de la universidad.

En segundo lugar podemos encontrar algunos \textit{chatbots} que permiten interactuar con Moodle de manera bastante similar al UBUVoiceAssistant. Entre los más destacables se encuentra \href{https://github.com/AnyTimeTraveler/moodlebot}{una integración con Telegram}, o \href{https://education.microsoft.com/en-us/resource/3dffb3a8}{el plugin para Microsoft Teams}, donde además del bot, encontramos la posibilidad de ver los cursos de manera visual en el cliente de mensajería. También podemos encontrar una \href{http://libres.uncg.edu/ir/asu/f/Melton_Michelle_2019_Thesis.pdf}{skill para Alexa}, que tiene un propósito y una funcionalidad parecida a nuestro proyecto.

También podemos encontrar una serie de asistentes que permiten interactuar con otras plataformas universitarias. Por ejemplo, tenemos \href{https://www.admithub.com/case-study/how-georgia-state-university-supports-every-student-with-personalized-text-messaging/}{Pounce} para interactuar con la \textit{Georgia State University}, o \href{https://www.youtube.com/watch?v=zsRPuU53E74}{Genie} en la \textit{Deakin University}. Es necesario mencionar el caso de \textit{Georgia Tech} \cite{georgiatech}, en el que, durante uno de sus cursos online se usaron dos \textit{chatbots} que sustituyeron a profesores de apoyo para responder algunas de las preguntas más habituales.

Finalmente, cabe destacar el proyecto de la \textit{Sant Louis University} \cite{alexauni}, donde instalaron un total de 2300 dispositivos \textit{Amazon Echo} con los que la comunidad universitaria puede hacer preguntas sobre la universidad a Alexa, sin necesidad de usar un dispositivo propio.

\begin{landscape}
Se puede ver una comparación de las características en la tabla \ref{tabla:CompCaracteristicas}

\tablaSmall{Comparación de características}{c c c c c c}{CompCaracteristicas}{
    Nombre del & Lenguaje de & Tipo de & Dispositivos & Complejidad de & Complejidad \\
    proyecto & programación & información & compatibles & instalación & de uso \\
}{
    UBUVoiceAssistant 2 & Python & Dinámica (Moodle) & Windows y Linux & Baja & Moderada\\
    UBUVoiceAssistant 1 & Python & Dinámica (Moodle) & Linux & Alta & Moderada \\
    UBUAssistant & Java & Estática & Android & Baja & Moderada \\
    Bot de Telegram & Java & Dinámica (Moodle) & PC y móvil & Alta & Alta \\
    MS Teams & PHP & Dinámica (Moodle) & PC y móvil & Alta & Baja \\
    Skill de Alexa & Java & Dinámica (Moodle) & Móvil & Moderada & Moderada \\
    Georgia State University & Desconocido & Dinámica & Móvil & Baja & Baja \\
    Deakin University & Desconocido & Estática & Móvil & Baja & Moderada \\
    Georgia Tech & Desconocido & Estática & Aula virtual & Ninguna & Baja\\
    Sant Louis University & Java & Estática & Amazon Echo/Dot & Ninguna & Moderada \\
}

\end{landscape}
\capitulo{7}{Conclusiones y Líneas de trabajo futuras}

\section{Conclusiones}

Una de las conclusiones que he sacado de este proyecto es que, aunque te encuentres con grandes problemas como fue el prototipo de skill para Alexa, en este tipo de proyectos es importante no abandonar y buscar otro camino, porque aunque parezca una pérdida de tiempo o algo problemático, puede ser que llegues a un mejor resultado que el esperado, como por ejemplo poder hacer una interfaz gráfica tras el cambio a Mycroft.

Otra conclusión a la que se ha llegado es que es importante dedicarle más tiempo a pensar qué es lo que quieres hacer que ponerse a hacer sin tener un plan, porque al final acabas dedicando más tiempo arreglando y cambiando cosas que el que dedicas cuando piensas las cosas.

Como conclusión final, se puede afirmar que se han cumplido los objetivos del proyecto. El producto final es una aplicación que permite al usuario interactuar con una plataforma de Moodle mediante voz y texto, configurar el asistente de voz y multitud más de funcionalidad.

Se han aplicado muchos de los conocimientos aprendidos durante el grado, como puede ser la interacción hombre-máquina, buenas prácticas de programación, defectos de diseño, refactorización de código, etc.

También se ha aprendido mucho en cuanto a la realización de interfaces gráficas, con las que no tenía apenas experiencia.

En definitiva, gracias a lo aprendido en la carrera se ha creado una aplicación funcional, útil y cumpliendo con los objetivos del proyecto. Ha sido una experiencia bastante agradable, aunque a veces te encuentres con problemas es muy satisfactorio solventarlos y te motiva para continuar con el proyecto y explorar un campo completamente desconocido para mí, como eran los asistentes de voz, es muy entretenido y se aprende mucho.

\section{Líneas de trabajo futuras}

Durante la realización del proyecto han surgido varias ideas sobre cómo se puede mejorar la aplicación.

Una de las mayores desventajas que tuvo cambiar de Alexa a Mycroft es que Mycroft no tiene soporte para Windows. Sin embargo, es posible ejecutar Mycroft desde Docker\cite{DockerSoftware2020}, que permite automatizar el despliegue de aplicaciones, por lo que se podría ejecutar la aplicación con Docker y utilizarla conectándose al contenedor desde Windows. Además también es posible utilizar Mycroft desde Android, que facilitaría mucho su uso para los estudiantes.

Otra posible continuación del desarrollo es guardar los datos que se obtienen de Moodle en ficheros locales de forma que se pueda utilizar la aplicación sin conexión a Internet. Usar la aplicación \textit{offline} reduciría bastante la funcionalidad de la aplicación, principalmente el servicio de STT, aunque gracias a que Mycroft es modular se pueden cambiar estos elementos. De hecho, el motor de TTS por defecto en Mycroft, Mimic\cite{Mimic} permite instalarse y utilizarse de forma local. En este \href{https://community.mycroft.ai/t/can-i-use-mycroft-offline/5306/5}{post del foro} de Mycroft se detalla bastante qué cambios son necesarios para utilizar Mycroft sin conexión.

Una de las formas más claras de mejorar la aplicación es aumentar la funcionalidad de la aplicación, permitiendo al usuario obtener más información de Moodle.

Por último se puede mejorar la interacción con el usuario. Esto implica mejorar la interfaz gráfica, que puede ser añadiendo funcionalidad o mejorar la claridad visual, y mejorar la interacción con Mycroft, de forma que las conversaciones sean más "humanas".


\printbibliography

\end{document}
