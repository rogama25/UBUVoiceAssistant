\capitulo{4}{Técnicas y herramientas}

\section{REST API}

Una API \cite{QueEsAPI}(\textit{Application Programming Interface}) es un conjunto de definiciones y protocolos usados para desarrollar e integrar el software de las aplicaciones. REST \cite{APIRESTQue} es una interfaz entre sistemas y que usa HTTP para obtener datos o realizar distintas operaciones sobre esos datos en todos los formatos posibles.

\section{Web Scraping}

\textit{Web scraping} \cite{QueEsWeb2019} es una técnica utilizada para obtener información de páginas web simulando la navegación de una persona en la web. Este recurso se utiliza cuando no hay otra opción para obtener datos de una página web. En este proyecto se implementó para obtener la sesión del usuario en UBUVirtual y conseguir datos que no podía obtener mediante otro medio. Finalmente esta técnica no se ha utilizado aunque se ha dejado implementada para posibles usos futuros.

\section{Moodle}

\href{https://moodle.org/}{Moodle} \cite{Moodle2020} es una plataforma de gestión de aprendizaje que permite a docentes y alumnos impartir y recibir clases a distancia, así como facilitar la gestión de los cursos. Es una herramienta de código abierto y por lo tanto cualquiera puede utilizarlo. En su página se pueden encontrar versiones de prueba como \href{https://school.moodledemo.net/}{Mount Orange School}

Es la plataforma en la que está basada UBUVirtual y es ampliamente utilizada en todo el mundo. Además posee una API con servicios web bastante completa que facilita mucho el trabajo.

\subsection{WebServices}

\textit{Web Services} es el nombre que se usa en Moodle para referirse a la API REST. Estos \textit{Web Services} facilitan la interacción con Moodle para los programadores, quienes no tienen que recurrir a otros métodos como el \textit{web scraping} que requieren un mayor tiempo para implementarlos. Los \textit{Web Services} basan su funcionamiento en un \textit{token}, que es una cadena de caracteres asociada a un usuario y que se obtiene a través de la función \textbf{moodle\_mobile\_app}. Gracias a este \textit{token} se puede acceder al resto de funciones del \textit{web service}, las cuales te permiten, por ejemplo, obtener los eventos del calendario, cursos de un usuario o sus notas.

\section{Metodología de desarrollo}

Para el desarrollo del proyecto se ha empleado metodología ágil, en concreto SCRUM. Se ha adoptado una estrategia de desarrollo iterativa, con \textit{sprints} normalmente de dos semanas en los que se han realizado las entregas parciales. Las reuniones de estado del proyecto se han realizado semanalmente, a diferencia de lo normal en SCRUM que se realizan diariamente.

\section{Mycroft}

Mycroft\cite{MycroftSoftware2019} es un asistente de voz de código abierto y software libre. Está disponible para sistemas operativos basados en Linux, Android, Raspberry Pi y dispositivos propios de Mycroft, como el \href{https://mycroft.ai/product/mycroft-mark-1/}{Mark 1}. Es una aplicación cliente que requiere una instalación previa, a diferencia de otros asistentes de voz que se ejecutan en la nube. Su diseño está pensado para que sea modular, permitiendo a los usuarios cambiar sus componentes.

Al ser de código abierto te permite explorar y cambiar su implementación y diseño, no como otros asistentes de voz que son una caja negra en la que no puedes ver nada. Al ser una aplicación cliente existen diferentes componentes, como se puede observar en la figura \ref{fig:arquitectura}, que tienen su propia responsabilidad dentro de la aplicación.

\subsection{Voice\cite{TechnologyOverview}}

Este componente es el encargado de transformar la voz a texto (\textit{speech-to-text} o \textit{STT}) y el texto a voz (\textit{text-to-speech} o \textit{TTS}).
Para el STT, Mycroft usa por defecto el motor de Google, ya que se necesita que la transformación sea rápida y precisa. Para añadir una capa adicional de privacidad, todas las peticiones de STT pasan por un proxy de los servidores de Mycroft. Así, Google no puede detectar si hay una persona haciendo miles de peticiones o son miles de personas haciendo pocas peticiones. Además del motor de Google, Mycroft permite usar otros motores de STT como Mozilla DeepSpeech o Kaldi.
En cuanto al TTS, se puede configurar desde la página del dispositivo. Las opciones de British Male y American Female usan Mimic 1, American Male usa Mimic 2 y Google Voice usa la voz de la API de Google Translate. Además de estos motores configurables desde la web, se puede usar Google TTS o Mycrosoft Azure, entre otros.

\subsection{Skills\cite{TechnologyOverview}}

Este componente usa principalmente un servicio, llamado servicio de intents (\textit{intent service}), que es el encargado de, dada una frase o \textit{utterance}, emparejarla con una \textit{skill}. Esto es posible gracias a los analizadores de intents (\textit{intent parser}) que usa Mycroft, Adapt y Padatious.
Adapt es una aplicación de código abierto ligera diseñada para usarse en dispositivos con recursos limitados, lo que es muy útil para Mycroft. Padatious es una aplicación basada en redes neuronales y \textit{machine learning} y es más efectiva y fácil de usar que Adapt, además de ofrecer más funcionalidad.

\subsection{MessageBus\cite{MessageBus}}

El MessageBus es el componente que permite que el resto de componentes se comuniquen entre sí. Es un \textit{websocket} que se encarga de pasar la información entre los componentes de Skills y Voice. Cuando el usuario dice una frase, el componente Voice lo transforma a texto mediante el \textit{TTS}, lo pasa al componente Skills mediante el MessageBus y éste decide qué \textit{Skill} hay que ejecutar. Una vez ejecutada la \textit{Skill}, se envía la respuesta al componente Voice, que la transforma a voz mediante el \textit{STT}.

\imagen{arquitectura}{Arquitectura de Mycroft}

\section{Python}

\href{https://www.python.org/}{Python} \cite{Python2020} es un lenguaje de programación interpretado, dinámico, multiplataforma y multiparadigma, soportando orientación a objetos, programación imperativa y programación funcional.

\section{GitHub}

\href{https://github.com/}{GitHub} \cite{GitHub2020} es una plataforma de desarrollo colaborativo para alojar proyectos usando el sistema de control de versiones Git. Ya que se ha empleado SCRUM se ha utilizado ZenHub, que es una herramienta para la administración de proyectos que se integra con GitHub y ofrece herramientas para la metodología agil como el \textit{Kanban} o distintos gráficos que te muestran la evolución del proyecto entre otras cosas.

\section{Atom}

\href{https://atom.io/}{Atom} \cite{AtomSoftware2020} es un editor de código fuente de código abierto multiplataforma desarrollado por GitHub. Tiene integrado control de versiones Git y se le pueden añadir \textit{plugins} para añadir distintas funciones.

\subsection{Plugins}

Los plugins que le he añadido a Atom son \href{https://atom.io/packages/atom-ide-ui}{atom-ide-ui} que mejora la UI de Atom y añade la funcion de IDE. Para que el IDE funcione con Python también se ha instalado el plugin \href{https://atom.io/packages/ide-python}{ide-python}.

\section{PyQt5}

\href{https://doc.qt.io/qtforpython/}{PyQt5} \cite{PyQt2020} es un \textit{binding} (una adaptación de una biblioteca para que sea usada en otro lenguaje) de Qt para Python. Qt es un framework multiplataforma orientado a objetos para desarrollar programas que utilizan interfaces gráficas de usuario. Es software libre y de código abierto.

\subsection{Qt Designer}
QtDesigner \cite{QtDesignerManual} es una herramienta para diseñar rápidamente interfaces gráficas a través de los \textit{Widgets} de Qt. Es muy útil para crear prototipos rápidos de lo que quieres hacer, mediante la funcionalidad de arrastrar y soltar (drag-and-drop) para poner los componentes de la interfaz y te permite traducir el prototipo creado a un lenguaje de programación como C++ o Python.

\imagen{qtdesigner}{Interfaz de Qt Designer}

\section{requests}

\href{https://requests.readthedocs.io/es/latest/}{Requests} es una biblioteca para Python que permite enviar peticiones HTTP. Ha sido extremadamente útil para este proyecto gracias a su simplicidad de uso y cantidad de funcionalidades que tiene.

\section{\LaTeX}

\LaTeX \cite{LaTeX2020} es un sistema de composición de textos que se utiliza para crear documentos con una alta calidad tipográfica. Se utiliza comúnmente en artículos y libros científicos. Para esta memoria se ha utilizado la distribución MiKTeX con el editor TeXstudio.

\subsection{MiKTeX}

\href{https://miktex.org/}{MiKTeX} es una distribución de \LaTeX que está siempre actualizada, es fácil de instalar e incluye muchos paquetes.

\subsection{TeXstudio}

\href{https://www.texstudio.org/}{TeXstudio} \cite{TeXstudio2019} es un editor de \LaTeX y un entorno de desarollo integrado (IDE), de código abierto. Ofrece varios servicios muy útiles, como resaltado de sintaxis o la corrección ortográfica. Es por esto que TeXstudio es una opción más atractiva que otros editores de \LaTeX.

\section{Análisis de software}

El análisis de software\cite{AnalisisSoftware2019} es un proceso mediante el cual se analiza el código fuente teniendo en cuenta propiedades como seguridad, robustez, fiabilidad o mantenibilidad. Es algo muy importante en cuanto al desarrollo software permitiendo al programador identificar posibles errores en el código. Para el proyecto se han utilizado las herramientas \href{https://app.codacy.com/}{Codacy} y \href{https://sonarcloud.io/}{SonarCloud}

\imagen{sonarcloud}{Resultados del análisis de SonarCloud}

\section{Zotero}

\href{https://www.zotero.org/}{Zotero} \cite{Zotero2020} es un gestor de referencias bibliográficas, de código libre y abierto y multiplataforma. Además de la versión cliente también dispone de una \href{https://zbib.org/}{página} para generar referencias bibliográficas. Ha resultado muy útil para generar la bibliografía en formato BibTeX y es muy cómodo de usar.

\section{PlantUML}

\href{https://plantuml.com/es/}{PlantUML} \cite{PlantUML2020} es una herramienta de código abierto que permite la creación de diagramas UML desde un lenguaje de texto plano. Con PlantUML solo tienes que especificar las relaciones que hay en el diagrama y te genera automáticamente el diagrama en una imagen.

\section{Amazon Web Services}

Los Amazon Web Services \cite{AmazonWebServices2020} son un conjunto de servicios de computación en la nube. En concreto, de todos estos servicios se utilizó Amazon Lambda para gestionar los eventos originados por la activación de la \textit{skill}. Amazon Lambda es una plataforma de \textit{serverless computing}\cite{ServerlessComputing2020}, que es un modelo de ejecución de computación en la nube en el que el proveedor proporciona el servidor y gestiona los recursos necesitados por el código ejecutado, lo que permite que el usuario pague únicamente por lo que necesita, en lugar de alquilar un servidor. Lambda ejecuta código como respuesta a eventos y es la base de los AWS. Al final no se han utilizado en el proyecto, pero se estudió su funcionamiento durante el desarrollo del prototipo de \textit{skill} para Alexa.