\apendice{Especificación de diseño}

\section{Introducción}

En este apartado se va a explicar como está diseñado el proyecto.

\section{Diseño de datos}

Los datos están correlacionados con el formato con el que se guardan en Moodle, aunque solo están modelados aquellos datos que me son útiles. En el paquete /src/model/ están las clases que modelan estos datos y son:
\begin{itemize}
	\item \textbf{User} para guardar datos relativos al usuario.
	\item \textbf{Course} para guardar los datos de los cursos.
	\item \textbf{Forum} para los foros de un curso.
	\item \textbf{Discussion} para las discusiones de un foro.
	\item \textbf{GradeItem} para las notas.
	\item \textbf{Event} para los eventos del calendario.
\end{itemize}
\imagen{diagrama_clases}{Diagrama de clases}
\section{Diseño procedimental}

A continuación se van a detallar las conexiones realizadas por la aplicación para realizar la conexión con Moodle y conseguir el token del usuario.

Lo primero que se hace es enviar una petición usando como parámetros en la URL el usuario y la contraseña y como respuesta se obtiene el token del usuario, necesario para realizar el resto de peticiones de los \textit{web services}

Después se obtiene información relevante para la aplicación a través del \textit{web service} \textbf{core\_webservice\_get\_site\_info}. La respuesta de esta petición devuelve el idioma de la plataforma de Moodle y la id del usuario de Moodle, que será necesaria para utilizar otros \textit{web services}

Por último se obtienen los cursos en los que está inscrito el usuario mediante el \textit{web service} \textbf{core\_enrol\_get\_users\_courses}
\imagen{diagrama_secuencia}{Diagrama de secuencia}
\section{Diseño arquitectónico}

El proyecto realmente consiste en dos aplicaciones, una la aplicación con la interfaz gráfica y la otra sería Mycroft. Ambas aplicaciones utilizan una arquitectura cliente-servidor, que es una arquitectura en la que el cliente realiza peticiones a través de un medio y el servidor envía una respuesta en función de la petición a través del mismo medio. La aplicación gráfica actúa como cliente y el servidor es la API REST de la plataforma de Moodle, es decir, los \textit{web services}. Mycroft tiene como cliente sus \textit{skills} que se comunican con dos servidores, uno para las funciones de STT y otro para TTS. Y entre las dos aplicaciones también existe una arquitectura cliente-servidor, siendo la aplicación gráfica el servidor y los clientes las diferentes \textit{skills}.
\imagen{diagrama_despliegue}{Diagrama de despliegue}