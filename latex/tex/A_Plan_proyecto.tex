\apendice{Plan de Proyecto Software}

\section{Introducción}

En este apartado se va a comentar la planificación temporal del proyecto así como la metodología que se ha seguido para su desarrollo y la viabilidad del proyecto en el marco económico y legal.

\section{Planificación temporal}

En el desarrollo se ha empleado SCRUM, que es un marco de trabajo para desarrollo ágil de software, pero en una versión adaptada a un equipo de 1 persona, a diferencia de los equipos normales que suelen ser de 5 a 9 personas. También se ha cambiado la frecuencia de las reuniones de seguimiento, siendo semanales en vez de diarias.

Estas reuniones se han utilizado para exponer el trabajo realizado, los problemas que se han encontrado durante su realización, mostrar el avance del proyecto y fijar nuevas tareas para continuar con el proyecto.

Como es normal utilizando metodología ágil el desarrollo ha sido iterativo organizado por \textit{sprints}. La duración de los \textit{sprints} ha sido de una, dos y tres semanas.

\subsection{Sprint 1}

El primer \textit{sprint} duró tres semanas (del 17 de febrero al 2 de marzo). La primera reunión de seguimiento se centró en explicar detalladamente los objetivos del trabajo. Gran parte del trabajo realizado en este \textit{sprint} se centró en investigar y aprender cómo funcionan los asistentes de voz, en concreto Alexa, así como centrarme en leer trabajos relacionados con lo que se pretendía en el proyecto para analizar qué cosas están bien hechas, qué se puede cambiar y la viabilidad de algunas otras cosas.

También aprendí cómo funciona Moodle, qué son los \textit{web services} y cómo utilizarlos. Finalmente, intentando hacer el prototipado de una primera versión de \textit{Skill} de Alexa, se decidió cambiar el asistente con el que se iba a desarrollar el trabajo.

\subsection{Sprint 2}

Este \textit{sprint} tuvo una duración de tres semanas (del 2 de marzo al 12 de marzo). En este \textit{sprint} me dediqué a plantear si seguir con este proyecto, cambiar totalmente o hacer alguna modificación. Finalmente me decanté por continuar con el asistente de voz pero necesitaba encontrar un asistente de voz que no me fuera a dar problemas. Se contempló el uso de Google Assistant pero no estaba seguro de que no me fuera a dar problemas similares que los que tuve con Alexa.

Continuando la búsqueda de un asistente me encontré con Mycroft, que era de código abierto y no tenía ningún tipo de limitación, además de estar bastante extendido para ser un asistente de este tipo. Así que decidí usar Mycroft para continuar el proyecto.
Tras comentarle la decisión al tutor, empleé el resto del \textit{sprint} para prototipar una primera version de la \textit{Skill} y contemplar nuevas opciones que se habían abierto al usar Mycroft en vez de Alexa, como un cliente gráfico.

\imagen{burndown2}{Gráfico \textit{burndown} del \textit{sprint} 2} \imagen{issues2}{\textit{Issues} completadas en el \textit{sprint} 2}

\subsection{Sprint 3}

El tercer \textit{sprint} duró una semana (del 18 al 25 de marzo) y se desarrolló una versión inicial de la interfaz gráfica. También se implementó el acceso al calendario mediante los \textit{web services} de Moodle. Finalmente estuve haciendo tests e investigando si podía haber algún tipo de problema en cuanto al número de accesos, y descubrí un problema de comunicación entre la aplicación y Mycroft.

\imagen{burndown3}{Gráfico \textit{burndown} del \textit{sprint} 3} \imagen{issues3}{\textit{Issues} completadas en el \textit{sprint} 3}

\subsection{Sprint 4}

En las dos semanas (del 2 al 16 de abril) que duró este \textit{sprint} seguí pensando sobre qué funcionalidad podía implementar en la \textit{skill} y se integró la opción de preguntar por los eventos de un día concreto, así como la posibilidad de realizar las preguntas vía texto en la aplicación. En este \textit{sprint} también se solucionó el problema de comunicación entre el proceso de la aplicación y los procesos de Mycroft mediante comunicación por \textit{sockets}.

\imagen{burndown4}{Gráfico \textit{burndown} del \textit{sprint} 4} \imagen{issues4}{\textit{Issues} completadas en el \textit{sprint} 4}

\subsection{Sprint 5}

El quinto \textit{sprint} fue de una semana (del 23 al 30 de abril) y se centró en aumentar las interacciones con la \textit{skill}, implementando las posibilidades de preguntar por actividad reciente dentro de un curso, eventos de un curso, notas finales del usuario y notas de un curso, que en esta primera implementación se utilizó \textit{web scraping} a falta de un \textit{web service}. Por último se mejoró la GUI permitiendo mostrar las respuestas también mediante texto y no solo audio.

\imagen{burndown5}{Gráfico \textit{burndown} del \textit{sprint} 5} \imagen{issues5}{\textit{Issues} completadas en el \textit{sprint} 5}

\subsection{Sprint 6}

Este \textit{sprint} tuvo una duración de dos semanas (del 30 de abril al 14 de mayo) y se centró en mejoras para la interacción del usuario con la aplicación. Entre estas mejoras están la recolección y muestra de \textit{logs}, tanto de Mycroft como de la app, comprobar si el micrófono está activo, y se decidió separar la \textit{skill} en tres para poder deshabilitar módulos. También se aumentaron las interacciones de la \textit{skill} permitiendo leer los foros de un curso.

\imagen{burndown6}{Gráfico \textit{burndown} del \textit{sprint} 6} \imagen{issues6}{\textit{Issues} completadas en el \textit{sprint} 6}

\subsection{Sprint 7}

El séptimo \textit{sprint} también duró dos semanas (del 14 al 28 de mayo) y se dedicó para mejorar la interfaz gráfica y crear el manual de instalación para que el tutor pudiera probarlo.

\imagen{burndown7}{Gráfico \textit{burndown} del \textit{sprint} 7} \imagen{issues7}{\textit{Issues} completadas en el \textit{sprint} 7}

\subsection{Sprint 8}

En las dos semanas que duró este \textit{sprint} (del 28 de mayo al 11 de junio) se realizaron cambios para mejorar la experiencia del usuario, como avisar si el idioma de la plataforma de Moodle a la que se intenta conectar es distinto al de la aplicación, internacionalizar la aplicación y una serie de mejoras visuales. Se cambió la interfaz gráfica implementando el uso de \textit{layouts} y se implementó la gestión de \textit{logs}. Por último se avanzó mucho en la memoria y anexos y se creó la primera \textit{release}.

\imagen{burndown8}{Gráfico \textit{burndown} del \textit{sprint} 8} \imagen{issues8}{\textit{Issues} completadas en el \textit{sprint} 8}

\subsection{Sprint 9}

En el noveno y último \textit{sprint} (del 11 al 25 de junio), se documentó y mejoró el código usando SonarCloud y se completó la memoria y anexos.

\imagen{burndown9}{Gráfico \textit{burndown} del \textit{sprint} 9} \imagen{issues9}{\textit{Issues} completadas en el \textit{sprint} 9}

\section{Estudio de viabilidad}

\subsection{Viabilidad económica}

El proyecto ha sido desarrollado por una persona, teniendo en cuenta que un programador junior cobra de media 18.695€\cite{SalariosParaEmpleos} de sueldo bruto anual y descontamos las cuotas del IRPF (2079,4€) y de la Seguridad Social (1.187,1€), nos queda que el sueldo neto anual son 15.428,5€
Así que, por los 4 meses de duración, el coste total del proyecto sería de 6.231,67€ y el programador cobraría 3.857,12€

Aunque el proyecto sea de código libre y abierto y la intención sea distribuirlo de forma gratuita, si se vendiera una copia a 1250 usuarios por 5€ la copia, se cubriría el coste total de desarrollo.

\subsection{Viabilidad legal}

El proyecto está licenciado bajo la licencia GPL-3.0. Esta licencia es muy permisiva, permitiendo usar, copiar, modificar, integrar con otro Software, publicar, sublicenciar o vender copias del Software a todo el que disponga de una copia y siempre y cuando todo el código fuente sea accesible y se utilice la misma licencia.

Las distintas herramientas que se han utilizado para el proyecto son:
\begin{itemize}
	\item Mycroft, que tiene una licencia Apache-2.0\cite{penrodHavingRightLicense2017}
	\item Bibliotecas de Python3:
	\begin{itemize}
		\item requests\cite{RequestsSoftware2020}, licencia Apache-2.0
		\item PyQt5\cite{RiverbankComputingLicense}, licencia GPL
	\end{itemize}
\end{itemize}

Los iconos utilizados en la aplicación se han sacado de \href{https://fontawesome.com/icons?d=gallery}{fontawesome}, que tienen licencia Creative Commons 4.0\cite{FontAwesome}