\capitulo{1}{Introducción}

Los asistentes de voz son una tecnología muy nueva y que no se ha utilizado en el campo de los sistemas de gestión de aprendizaje, aún teniendo unas características que los hacen muy interesantes para el ámbito de la educación. La posibilidad de obtener información importante para el estudiante a través de un comando de voz facilita mucho la interacción con este tipo de plataformas de educación.

Este proyecto se basa en la utilización de un asistente de voz para realizar diferentes tareas en una plataforma de Moodle, como es  UBUVirtual. Estas tareas son recuperar los eventos del calendario de un curso, de un día en concreto o en las próximas dos semanas, leer los foros de un curso y recuperar las notas de un usuario tanto finales como de un curso en concreto. La utilización de estas nuevas tecnologías permiten obtener información de la plataforma de Moodle de una forma más interactiva y cómoda, sin tener que navegar por la web.

En esta memoria se van a ir explicando de forma ordenada los conceptos necesarios para entender el proyecto, las técnicas y herramientas utilizadas para su realización y los problemas y decisiones que se han surgido durante el desarrollo del mismo.